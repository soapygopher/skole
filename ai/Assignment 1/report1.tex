\documentclass[letterpaper, 12pt]{article}

\usepackage[english]{custompack}
\usepackage{fullpage}
%\usepackage{hyperref}
%\usepackage{abcproblem}
\usepackage{listings}
%\renewcommand{\thesubsection}{\alph{subsection}}

%\allowdisplaybreaks
\numberwithin{theorem}{section}

% (fold)
\lstset{ 
	basicstyle=\ttfamily\footnotesize,
	keywordstyle=\color{blue}, 
	commentstyle=\color{gray}, 
	stringstyle=\color{darkgreen}, 
	emphstyle=\color{purple},
	%identifierstyle=, %vanlige ting
	numbers=left, 
	numberstyle=\footnotesize\ttfamily\color{gray}, 
	numbersep=1em,
	stepnumber=1, 
	%firstnumber=last, % dersom listingen deles i to, forsett der man slapp
	firstnumber=1,
	showspaces=false, 
	showstringspaces=false,
	captionpos=b,
	%frame=single,
	breakatwhitespace,
	breaklines=true, 
	tabsize=4,
	% python-spesifikt: 
	language=python,
	morekeywords={None, True, False, with, as, yield, self}, 
	emph={tuple, int, append, raw_input, upper, map, reduce, filter, len, str, open, hex, bin, format, range, xrange, title, strip, split, pop, join, min, max, sum, chr, set, list, sort, sorted, reversed, enumerate, ValueError, TypeError, IndexError, dict, zip, ord, any, all, count, Exception, __init__, __str__, __repr__, __class__, __name__, KeyboardInterrupt, hash, items, add, lower, upper, float}
}
% (end)

\title{\textbf{Assignment 1}}
\author{Håkon Mork \\ ESCE 526 Artificial Intelligence}
\date{\today}

\begin{document}
\maketitle
\noindent

\section{}
\subsection{Number of states visited}
Game A:
\begin{table}[h]
	\centering
	\small
	\begin{tabular}{lrrrr}
		Cutoff depth & 3 & 4 & 5 & 6 \\
		\midrule
		Minmax & 77,445 & 1,276,689 & & \\
		$\alpha$-$\beta$ pruning & 4,129 & 48,203 && \\
		\midrule
		Improvement & $\times$18.76 & $\times$26.49
	\end{tabular}
\end{table}

\noindent Game B:
\begin{table}[h]
	\centering
	\small
	\begin{tabular}{lrrrr}
		Cutoff depth & 3 & 4 & 5 & 6 \\
		\midrule
		Minmax & & & & \\
		$\alpha$-$\beta$ pruning & & & & \\
		\midrule
		Improvement & & & &
	\end{tabular}
\end{table}

\noindent Game C:
\begin{table}[h]
	\centering
	\small
	\begin{tabular}{lrrrr}
		Cutoff depth & 3 & 4 & 5 & 6 \\
		\midrule
		Minmax & & & & \\
		$\alpha$-$\beta$ pruning & & & & \\
		\midrule
		Improvement & & & &
	\end{tabular}
\end{table}


\subsection{Does state generation order matter?}
My evaluation function iterates through the successor states in the order they were generated: left-to-right, top-to-bottom, with the directions generated in the (arbitrary) order north-east-south-west. I shuffled\footnote{I used the Python standard library's \texttt{random.shuffle} function, which shuffles a sequence (e.g., our list of successor states) in place.} the list of possible successors before evaluating their subtrees, which should indicate whether there were any serious discrepancies in the number of states visited. 

\subsection{Delaying defeat}

\section{}
\subsection{Choice of evaluation function}

\subsection{Number of nodes visited}

\subsection{Tradeoff between evaluation function and game tree depth}

%\subsection{Gameplay log}


\clearpage
\appendix
\section{Appendix: Source code}
\subsection{Implementation comments}
I consider the successors of the current state to be on cutoff level 0, and pass a depth parameter to the alpha-beta and minmax algorithms. This depth parameter is incremented for each ply in the recursion tree until the given cutoff limit is hit, at which point the heuristic evaluation of that state is returned.

\subsection{Listing}
\lstinputlisting{./ass1.py}

\end{document}
